\frame{
  \frametitle{Procedure}  
  The main procedure is based on the repeated application of the rules according to the following steps,
  \begin{enumerate}
    \item $Check\ conflicts$ 
    \item $Propagate$
    \item $Split\ by Length$
    \item $Normalize$
    \item $Partition$
    \item $Check\ cardinality$
  \end{enumerate}
  
  Note:
    \begin{itemize}
      \item $S:$ set of string constraints.
      \item $A:$ set of arithmetic constraints.
    \end{itemize}
  
}

\frame{
  \frametitle{Procedure: Step 0:$Start\ of\ the\ procedure$}

\begin{textblock*}{41mm}(100mm,-5mm)
\tiny
\begin{exampleblock}{}
  	\begin{enumerate}
  	\setlength\itemsep{0.1em}
  	        \item $Check\ conflicts$ 
  	        \item $Propagate$
  	        \item $Split\ by\ Length$
  	        \item $Normalize$
  	        \item $Partition$
  	        \item $Check\ cardinality$  	    
  	\end{enumerate}
\end{exampleblock}
\end{textblock*}

  \begin{itemize}
    \item This is the start of the procedure. \begin{figure}[htb]
\begin{minipage}{.45\linewidth}
\[
 \texttt{con}(\mathbf{s},  \texttt{con}(\mathbf{t}), \mathbf{u}) \rightarrow   \texttt{con}(\mathbf{s}, \mathbf{t}, \mathbf{u})
\]
\[
\texttt{con}(\mathbf{s}, \epsilon, \mathbf{u}) \rightarrow \texttt{con}(\mathbf{s}, \mathbf{u})
\]
\[
 \texttt{con}(s) \rightarrow s
\]
\[
 \texttt{con}() \rightarrow \epsilon
\]
\end{minipage}%
\begin{minipage}{.45\linewidth}
\[
 \texttt{con}(\mathbf{s}, c_1 \cdots c_i, c_{i+1} \cdots c_n, \mathbf{u}) \rightarrow \texttt{con}(\mathbf{s}, c_1 \cdots c_n, \mathbf{u})
\]
\[
 \texttt{len}( \texttt{con}(s_1,\cdots,s_n)) \rightarrow \texttt{len}(s_1) + \cdots + \texttt{len}(s_n)
\]
\[
  \texttt{len}(c_1,\cdots,c_n) \rightarrow n
\]
\end{minipage}
\caption{Normalization rewrite rules for terms. The rules are taken from \cite{main-paper}}
\label{fig:normalization_rewrite_rules}
\end{figure}
    \item All the terms must be in normalized form, according to the above reduction rules.
  \end{itemize}
        \begin{description}
          \item e.g. $\texttt{con}("ab", \texttt{con}("c", l)) \rightarrow \texttt{con}("abc",l) $
          \item e.g. $\texttt{len}("abc") \rightarrow 3$
          \item e.g. $\texttt{len}(\texttt{con}("abc", "def"))\rightarrow 6$
        \end{description}
}

\frame{
  \frametitle{Procedure: Step 1:$Check\ conflicts$}

\begin{textblock*}{41mm}(100mm,-5mm)
\tiny
\begin{exampleblock}{}
  	\begin{enumerate}
  	\setlength\itemsep{0.1em}
  	        \item \underline{$Check\ conflicts$} 
  	        \item $Propagate$
  	        \item $Split\ by\ Length$
  	        \item $Normalize$
  	        \item $Partition$
  	        \item $Check\ cardinality$  	    
  	\end{enumerate}
\end{exampleblock}
\end{textblock*}

  \begin{itemize}
    \item Check is there any contradiction exists in the current \\constraints
    \item Rules used: \texttt{S-Conflict} , \texttt{A-Conflict}.
    \begin{figure}
\begin{align*}
 \texttt{A-Conflict} &\frac{ A \models_{LIA} \perp }{ \textnormal{unsat}}\\  
 \texttt{S-Conflict} &\frac{ s\approx t \in S \ \ s \not\approx t \in S}{\textnormal{unsat}}
\end{align*}
\end{figure}
    \item Examples: 
      \begin{itemize}
          \item e.g. $S:\{ \cdots, x \approx \epsilon,x\not\approx \epsilon, \cdots\} \rightarrow{\texttt{unsat}}$      
          \item e.g. $S:\{ \cdots, x \approx y,x\not\approx y, \cdots\} \rightarrow{\texttt{unsat}}$      
          \item e.g. $A:\{ \texttt{len}(x) \approx \texttt{len}(y), \texttt{len}(x) \not\approx \texttt{len}(y) \}\rightarrow{\texttt{unsat}}$      
       \end{itemize} 
  \end{itemize}
}

\frame{
  \frametitle{Procedure: Step 2:$Propagate$}

\begin{textblock*}{41mm}(100mm,-5mm)
\tiny
\begin{exampleblock}{}
  	\begin{enumerate}
  	\setlength\itemsep{0.1em}
  	        \item $Check\ conflicts$ 
  	        \item \underline{$Propagate$}
  	        \item $Split\ by\ Length$
  	        \item $Normalize$
  	        \item $Partition$
  	        \item $Check\ cardinality$  	    
  	\end{enumerate}
\end{exampleblock}
\end{textblock*}

  \begin{itemize}
    \item Introduce new constraints induced by constraints of other theories \\(e.g. $\mathcal{T}_{LIA}$ and $\mathcal{T}_{SL} $).  
    \item Rules used : \texttt{S-Prop}, \texttt{A-Prop}.
    \begin{figure}
    \begin{align*}
 \texttt{A-Prop} &\frac{ S \models  \texttt{len} \ x \approx \texttt{len} \ y}{ A := A, \texttt{len} \ x \approx \texttt{len} \ y}\\
  \texttt{S-Prop} &\frac{ A \models_{LIA}  \texttt{len} \ x \approx \texttt{len} \ y}{ S := S, \texttt{len} \ x \approx \texttt{len} \ y}
      \end{align*}
    \end{figure}    
    
    \item Examples: 
      \begin{itemize}
          \item e.g. $S \models (\texttt{len} \ x \approx \texttt{len} \ y) \rightarrow A:=A,{\texttt{len} \ x \approx \texttt{len} \ y}$      
          \item e.g. $S \models (\texttt{len} \ x \approx \texttt{len} \ "abc") \rightarrow A:=A,{\texttt{len} \ x = 3 }$      
       \end{itemize} 
  \end{itemize}
}

\frame{
	\frametitle{Procedure: Step 3:$Split\ by\ Length$}
	
	\begin{textblock*}{41mm}(100mm,-5mm)
	\tiny
	\begin{exampleblock}{}
	  	\begin{enumerate}
	  	\setlength\itemsep{0.1em}
	  	        \item $Check\ conflicts$ 
	  	        \item $Propagate$
	  	        \item \underline{$Split\ by\ Length$}
	  	        \item $Normalize$
	  	        \item $Partition$
	  	        \item $Check\ cardinality$  	    
	  	\end{enumerate}
	\end{exampleblock}
	\end{textblock*}
	
	\begin{itemize}
     	\item Introduce new constraint into the set of arithmetic \\constraints for equalities in string constraints.
     	\item Introduce branching for free variables in string constraints.
		\item Rules used: \texttt{Len} and \texttt{Len-Split}.
		
		\begin{figure}
			\begin{align*} 
			\texttt{Len} &\frac{ x \approx t \in \mathcal{C}(S) \ x \in \mathcal{V}(S)}{ A := A, \texttt{len} \ x \approx (\texttt{len} \ t) \downarrow}\\
			%\texttt{Len-Split} &\frac{ x \in \mathcal{V}(S \cup A) \ x: Str}{ S := S, x \approx \epsilon \parallel  A:= A, \texttt{len} \ x > 0}
			\end{align*}
		\end{figure}    
		
		\item Examples: 
		\begin{itemize}
			\item e.g. $S \models (x \approx y) \rightarrow A:=A,{\texttt{len} \ x \approx \texttt{len} \ y}$      
			\item e.g. $S \models (x \approx "abc") \rightarrow A:=A,{\texttt{len} \ x = 3 }$      
		\end{itemize} 
	\end{itemize}
}


\frame{
	\frametitle{Procedure: Step 4:$Normalize$}
	
	\begin{textblock*}{41mm}(100mm,-5mm)
	\tiny
	\begin{exampleblock}{}
	  	\begin{enumerate}
	  	\setlength\itemsep{0.1em}
	  	        \item $Check\ conflicts$ 
	  	        \item $Propagate$
	  	        \item $Split\ by\ Length$
	  	        \item \underline{$Normalize$}
	  	        \item $Partition$
	  	        \item $Check\ cardinality$  	    
	  	\end{enumerate}
	\end{exampleblock}
	\end{textblock*}
	
	\begin{itemize}
		\item Compute the normalized form for each term. 
		\item If there is term not in normalized, apply splitting and then finally unify. 
		\item Rules used: \texttt{S-Cycle}, \texttt{L-Split}, \texttt{F-Unify}		
		\begin{figure}
			\scriptsize
			\begin{align*}
			\texttt{F-Unify} &\frac
			{ F \ s = ( w,u,u_1) \ \ F \ t = (w, u,v_1) \  s\approx t \in \mathcal{C}(S) \ S \models \textnormal{len} \ u \approx \textnormal{len} \ v }
			{ S:= S, u \approx v}
			\end{align*}
		\end{figure}    
		
		\item Examples: 
		\begin{itemize}
			\item e.g. $\texttt{con}(x, m) \approx \texttt{con}(y, n) , \texttt{len}(x) \approx \texttt{len}(y)  \rightarrow S:=S,{ x \approx  y}$
		\end{itemize} 
	\end{itemize}
}


\frame{
	\frametitle{Procedure: Step 5:$Partition$}
		\begin{textblock*}{41mm}(100mm,-5mm)
		\tiny
		\begin{exampleblock}{}
		  	\begin{enumerate}
		  	\setlength\itemsep{0.1em}
		  	        \item $Check\ conflicts$ 
		  	        \item $Propagate$
		  	        \item $Split\ by\ Length$
		  	        \item $Normalize$
		  	        \item \underline{$Partition$}
		  	        \item $Check\ cardinality$  	    
		  	\end{enumerate}
		\end{exampleblock}
		\end{textblock*}
	
	
	\begin{itemize}
		\item Each equivalence class should be in their corresponding group (bucket).
		\item If the partition is not compete, apply splitting on the free variables.
		\item Rules used: \texttt{D-Base}, \texttt{D-Add}, \texttt{S-Split} and \texttt{L-Split}. 		
		\begin{figure}
			\scriptsize
			\begin{align*}
			            \texttt{S-Split} &\frac
						{ x, y \in \mathcal{V}(S) \ \ x \approx y, x \not\approx y \in \mathcal{C}(S) }
						{ S := S, x \approx y \parallel S:= S, x \not\approx y}\\
						\texttt{L-Split} &\frac
						{ x, y \in \mathcal{V}(S) \ \ x,y: \textnormal{Str}  \ \  S \not\models \texttt{len} \ x \not\approx \textnormal{len} \ y}
						{ S:= S, \textnormal{len} \ x \approx \texttt{len} \ y \parallel S:=S, \textnormal{len} \ x \not\approx \texttt{len} \ y}
			\end{align*}
		\end{figure}   		
	\end{itemize}
}
\frame{
	\frametitle{Procedure: Step 6:$Check\ cardinality$}
	
		\begin{textblock*}{41mm}(100mm,-5mm)
		\tiny
		\begin{exampleblock}{}
		  	\begin{enumerate}
		  	\setlength\itemsep{0.1em}
		  	        \item $Check\ conflicts$ 
		  	        \item $Propagate$
		  	        \item $Split\ by\ Length$
		  	        \item $Normalize$
		  	        \item $Partition$
		  	        \item \underline{$Check\ cardinality$}  	    
		  	\end{enumerate}
		\end{exampleblock}
		\end{textblock*}
	
	\begin{itemize}
		\item For each bucket $B$ introduce new an arithmetic constraint, as the alphabet $\mathcal{A}$ is finite.
		\item Performed as a last step of the procedure.		
		\item For example, 
		 \begin{itemize}
		 	\item If
		    \begin{itemize}
				\item we have 256 characters, and
				\item $S$ entails that 257 distinct strings of length 1 exist 
		    \end{itemize}
		    \item Then \begin{itemize}
				\item $S$ is unsatisfiable
			\end{itemize}
		\end{itemize}	  

	\end{itemize}
}