\subsection{Examples}
Now we try to illustrate the procedure's workings with respect of two example. First example shows how the procedure conclude $unsat$ for a set of constraints which are unsatisfiable. And the second shows how the procedure finds a saturated configuration for a set of constraints which are satisfiable. Both the examples are taken from \cite{main-paper}.
\begin{description}			
	\item[Example 1:] The input constraints are: $ A = \phi$ and $S = \{ \texttt{len}(x) \approx
	\texttt{len}(y), x \not\approx \epsilon, y\not\approx \epsilon, z \not\approx \epsilon, \texttt{con}(x, l_1, z) \approx \texttt{con}(y, l_2, z)\}$, where $l_1$, $l_2$ are distinct constants of the same length.	

The procedure starts with $step\ 0$ by applying \texttt{Reset}. It tries to find conflicts by applying   \texttt{S-Conflict} and \texttt{A-Conflict}. Since the arithmetic constraint set $A$ is empty and there are no contradicting string constraints in $S$, $check\ for\ contradiction$ fails. 

The procedure tries the step $propagate$. For this example this step does not introduce any new constraint in $S$, as $A$ is empty. Now the procedure applies \texttt{Len} and \texttt{Len-Split} repeatedly as long as there is possible application. The application of these two splitting rules causes the branching of the derivation tree. However, all branches are closed by rule  \texttt{S-Conflict} expect one. This is shows in Figure as the left branches of each configuration. 

In this non closing configuration, the string equivalence classes are $\{x\}, \{y\}, \{z\}, \{l_1\}, \{l_2\}, \{\epsilon\},$ and $\{ \texttt{con}(x, l_1, z), \texttt{con}(y, l_2, z)\}$. Now the procedure goes on to compute normal forms by applying \texttt{N-Form2}, \texttt{F-Form2} and \texttt{N-Form1}.The flat forms  \texttt{F} $\texttt{con}(x, l_1, z) = (x, l_1, z)$ and \texttt{F} $\texttt{con}(y, l_2, z) = (y, l_2, z)$ are computed by \texttt{F-Form1}. Then \texttt{F-Unify} is applied to add the equality $x \approx y$ to $S$. This causes the procedure to restart, but with an extended constraints set. Which induces a new equivalence classes $ \{x, y\}, \{z\}, \{l_1\}, \{l_2\}, \{\epsilon\}$, and $\{\texttt{con}(x, l_1, z), \texttt{con}(y, l_2, z)\}$. After similar steps, the procedure reached a stage where it computes the flat forms $(x, l_1, z)$ and $(x, l_2, z)$ for the  corresponding equivalence class,
by choosing $x$ as representative of $y$. At this point the procedure applies rule \texttt{F-Unify} again. This introduces a new equality $l_1 \approx l_2$ to $S$ and eventually derives \texttt{unsat} with \texttt{S-Conflict}. That is the procedure ended up with a derivation tree where each branch is closed and conclude that the input constraints are unsatisfiable.  The resulting derivation tree is shown in figure .

\input{chapter/example_1_tree.tex}

	
	
\end{description}
\begin{description}			
	\item[Example 2:] The input constraints are: $ A = \phi$ and $S = \{ \texttt{len}(x) \approx
	\texttt{len}(y), x \not\approx \epsilon, z \not\approx \epsilon, \texttt{con}(x, l_1, z) \not\approx \texttt{con}(y, l_2, z)\}$, with $l_1$, $l_2$ are distinct constants of the same length.
	
	
	Similar to the previous example, the procedure reaches a configuration where the string equivalence classes are $ \{x\}, \{y\}, \{z\}, \{l_1\}, \{l_2\}, \{\epsilon\},\{ \texttt{con}(x, l_1, z)\}$ and,$\{\texttt{con}(y, l_2, z)\}$.
	In this case, the procedure attempts to partition the the equivalence classes into buckets. It fails to create the full partition, as rule \texttt{D-Base}  nor rule \texttt{D-Add} is applicable to $[y]$ because of $S \models \texttt{len} \ x \approx \texttt{len} \ y$ and  $x \not\approx y \not\in \mathcal{C}(S)$.  
	
	In such a case the procedure tries to guess by applying rule \texttt{S-Split} to $x$ and $y$. It case two branches in the derivation tree, one for $x \approx y$ and another for $x \not\approx y$.  
		
	After similar steps as in previous example, the procedure can derive a configuration where the string equivalence classes are $ \{x\}, \{y\}, \{z\}, \{l_1\}, \{l_2\}, \{\epsilon\},\{ \texttt{con}(x, l_1, z)\}$ and,$\{\texttt{con}(y, l_2, z)\}$. After computing normal forms for these classes, it attempts to	construct a partition \texttt{B} of them into buckets. However, notice that if it adds {[x]},say, to \texttt{B} using \texttt{D-Base}, then neither \texttt{D-Base} (since $S \models \texttt{len} \ x \approx \texttt{len} \ y$) nor \texttt{D-Add} (since $x \not\approx y \not\in \mathcal{C}(S)$ ) is applicable to $[y]$. So it applies \texttt{S-Split} to $x$ and $y$. In the branch where $ x \approx y$, the procedure subsequently restarts as new constraints are introduced to $S$. Eventually the procedure computes normal forms and succeeds in making a full partition of \texttt{B}. The procedure places $[\texttt{con}(x, l_1, z)]$ and $[\texttt{con}(y, l_2, z)]$ into the same bucket using \texttt{D-Add}, which applies because their corresponding normal forms are $(x, l_1, z)$ and $(x, l_2, z)$ respectively. Any further rule applications lead to branches with  a saturated configuration. Thus the procedure concludes that the problem is satisfiable. For each of the branches with saturated configuration satisfying models can be generated.
\end{description}
Here in this section we have tried to explore the inner working of the decision procedure. According to the authors \cite{main-paper}, the presented version the procedure is a simplified one. The procedure and derivation rules, which are implemented as part of cvc4 is much more complex and elaborate.