\section{Example}
In this section, we will use the example formula \ref{mainConstraint} to explain how the decision procedure works. Consider the five constraints $s_1,s_2, s_3,s_4,s_5$ on the left side of Figure \ref{smt_example_1_2}. Since the SAT core cannot interpret the constraints, it treats them as five independent boolean variables and tries to assign values to them. The core starts by setting $s_1, s_4$ and $s_5$ to \texttt{true}. Then the SAT core choose true for $s_2$ before $s_3$.  Now the selected constraints $\{ s_1,s_2, s_4,s_5\}$  are forwarded to the theory solver. This is shown as the left branch of the tree on the right side of Figure \ref{smt_example_1_2}. The theory solver will answer \texttt{unsat} for this set of constraints. The SAT core backtracks and tries the other option. That is the SAT core choose $s_3$ and ask theory solver to solve the new set of constraints  $\{ s_1,s_3, s_4,s_5\}$. This is shown as the right branch of the tree on the right of Figure \ref{smt_example_1_2}. The theory solver eventually finds a satisfying solution in this branch. 

\begin{description}
	\item[Left derivation tree:] The input constraints are: $A = \{ \texttt{len}\ x > 6\}$ and $S= \{ x = \texttt{con}('ab', z), y = \texttt{con}('de', z), x = y\}$.
		
	The procedure starts with by applying \texttt{Reset}. Then it tries to find conflict by applying \texttt{S-Conflict} and \texttt{A-Conflict} and fails to find any contradiction in the current constraints. 
	
	The procedure tries to apply \texttt{A-Prop} and \texttt{S-Prop}. The constraints in different theories fail to induce new constraint for other. 
	
	Now the procedure keeps trying to apply rules \texttt{Len} and \texttt{Len-Split}. The application of these split rule causes the tree to branch into new configurations. However, all the branches are closed with respect to the rule \texttt{S-Conflict}. This is shown in the left branches in Figure \ref{leftDerivationTree}.
	
	Now for the non closing configuration, the procedure goes on to compute the normal forms. $x = y$ causes  $\texttt{con} ('ab', z) ,\texttt{con} ('de', z)$ to be classified as equivalent term. Further application of rule  \texttt{F-Unify} causes to introduce $'ab' = 'de'$ into $S$ and eventually this branch get closed by \texttt{S-Conflict}. Thus the procedure decides that, the given set of constraints are unsatisfiable.
	
	
	\item[Right derivation tree:] The input constraints are: $A = \{ \texttt{len}\ x > 6\}$ and $S= \{ x = \texttt{con}('ab', z), y = \texttt{con}('abc', l), x = y\}$. The derivation tree is shown in Figure \ref{rightDerivationTree}.
		
	Similar to the left derivation tree, the procedure reaches a configuration, where the string equivalence classes are like  $\{ x,y, \texttt{con}('ab', z), \texttt{con}('abc', l)\}, \{ z\}, \{ l\}$. Application of the rule \texttt{F-Unify} causes the introduction of new constraints $ z = \texttt{con}('c', l)$. The procedure restarts with new equivalence classes  $\{ x,y, \texttt{con}('ab', z), \texttt{con}('abc', l)\}, \{ z,  \texttt{con}('c', l)\}, \{ l\}$ with the application of \texttt{Reset}. The new constraint in $S$ induces new constraints in $A$. That application of \texttt{Len} causes $A: =A , \texttt{len} \ z = 1 + \texttt{len} \ l$. At this point the procedure starts to compute the normal forms by applying different rules. Eventually the constraint $\texttt{len}\ l > 3$ is computed and a saturated configuration is reached. As no more rule application is not possible at this configuration. Thus the procedure concludes that the problem is satisfiable. 
	
	Finally, for each buckets an arithmetic constraint on the minimum length is added. For example, $\texttt{len}\ l > 3$ causes the length of the bucket which contain $l$ to be 4. Now it is trivial to find a satisfying assignment. that if $ l = 'AAAA'$ then $ z = 'cAAAA'$ and $x, y = 'abcAAAA'$.
	
\end{description}