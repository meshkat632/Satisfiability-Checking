\section{Problem Definition}
\label{sec:problem_definition}
Introduces terms used in your topic by definitions. Furthermore, it 
can introduce theorems on which parts of your topic base. Hint: Use 
paragraphs to structure your text. This is the first paragraph.

And this is the second paragraph. By the way: Do not use 
abbreviations as don't, it's, or can't. In Figure~\ref{fig:graph} 
you can see an example of a picture embedded in a figure. The
picture is created using the \TikZ-Library (cf. 
\href{../manuals/tikzpgfmanual.pdf}{\TikZ-Manual}). In Table
\ref{tab:nameOfTheTable} you can see an example for a table.

\begin{definition}[Name of the term]
\label{def:nameOfTerm}
This is how you define a term.
\end{definition}

\begin{theorem}[Name of the theorem]
\label{the:nameOfTheorem}
This is how you write a theorem. Do not forget to prove the theorem.
\begin{proof}
	Here you write the proof of the theorem.
\end{proof}
\end{theorem}

\begin{figure}[htb] % where to insert the figure: h=here, t=top, b=bottom,
					% the order htb shows which position is preffered
	\begin{center}
		\begin{tabular}{cc}
			\begin{minipage}{0.35\linewidth} % minipages are a nice
											 % equipment to arrange
											 % pictures
				\begin{center}
					\input{pictures/directedGraph.tex}
				\end{center}
			\end{minipage}
			&
			\begin{minipage}{0.55\linewidth} % note that the total
											 % width of the mini
											 % pages is less than
											 % 1.0\linewidth, since
											 % otherwise you get a
											 % badbox (try it yourself)
				\begin{center}
					\input{pictures/directedTree.tex}
				\end{center}
			\end{minipage}
		\end{tabular}
	\end{center}
	\caption{A digraph on the left and a directed tree on the right.}
	\label{fig:graph}
\end{figure}

In the next lines you can see some examples formulas and other
constructs, which are useful in the math mode. A very useful webpage
to find symbols and the packages to include is
You can use the math mode in the text, e.g. $1\neq0$, or write it in
a whole line:
\[\left|\begin{array}{ccc}
		a_{1,1} & \ldots & a_{1,n}  \\
				& \vdots & 			\\
		a_{n,1} & \ldots & a_{n,n}
	\end{array}
\right|
\quad = \quad 
\left\{
	\begin{array}{ll}
			{\displaystyle \sum\limits_{\sigma\in S_n}} 
			% displaystyle avoids that symbols size get adjusted
			\left(
				\text{sgn}(\sigma)
				\prod\limits_{i=1}^{n}a_{i,\sigma(i)}
			\right) 
			& 
			\text{, if }\True 
		\\[0.6cm] % adjust the line spacing
			{\displaystyle\frac{42}{1}} & \text{, otherwise}
	\end{array}
\right.\]


\begin{table}[htb]
	\caption{This a a table.}
	\label{tab:nameOfTheTable}
	\bigskip % Insert a vertical space. Also \smallskip or \medskip.
	\begin{center}
		\begin{tabular}{|l||l|c|r|}
			\hline
			& align left 	& centered & align right \\
			\hline \hline
			row $1$ & box $1.1$ & box $1.2$ & box $1.3$ \\
			\hline
			row $2$ & \multicolumn{2}{c|}{box $2.1$} & box $2.2$ \\
			\hline
			row $2$ & \multicolumn{3}{c|}{box $2$} \\
			\hline
		\end{tabular}
	\end{center}
\end{table}

