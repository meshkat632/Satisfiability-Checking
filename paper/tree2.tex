% Author: Till Tantau
% Source: The PGF/TikZ manual

\documentclass{article}

\usepackage{pgf}
\usepackage{tikz}
\usetikzlibrary{arrows,automata}
\usepackage[latin1]{inputenc}
\usepackage{verbatim}

\begin{document}
	
	\begin{comment}
	:Title: State machine
	:Tags: Manual, Automata, Graphs
	
	Another examle from the manual.
	
	| Author: Till Tantau
	| Source: The PGF/TikZ manual
	
	\end{comment}
	
	\begin{tikzpicture}[->,>=stealth',shorten >=1pt,auto,node distance=2.5cm,semithick]
	\tikzstyle{every state}=[shape= rectangle,fill=lightgray,draw=none,text=black]
	
	\node[state]         (A)                                      {$initial\ \mathcal{K}$};
	\node[state]         (B) [below left of=A]            {$\texttt{S:= S},x\approx\epsilon$};
	\node[state]         (B_unsat) [below left of=B]  {$\texttt{unsat}$};
	\node[state]         (C) [below right of=A]         {$\texttt{A:= A},\texttt{len} \ x > 0$};
	\node[state]         (C_unsat) [below left of=C]  {$\texttt{unsat}$};
	\node[state]         (D) [below right of=C]    {$D$};	
	
	\path (A) edge              node {$\texttt{Len-Split}$} (B)
	              edge              node {$\texttt{S-Conflict}$} (C)
	         
	        (C) edge              node {$\texttt{S-Conflict}$} (C_unsat)
	             edge              node {$\texttt{Len-Split}$} (E)
			(B) edge              node {$\texttt{S-Conflict}$} (B_unsat)
	;
	\end{tikzpicture}
	
\end{document}