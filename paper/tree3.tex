% Rule based diagram
% Author: Remus Mihail Prunescu
\documentclass{minimal}
\usepackage[a4paper,margin=1cm,landscape]{geometry}
\usepackage{tikz}

%%%<
\usepackage{verbatim}
\usepackage[active,tightpage]{preview}
\PreviewEnvironment{tikzpicture}
\setlength\PreviewBorder{5pt}%
%%%>

\begin{comment}
:Title:  Rule based diagram

\end{comment}

\usetikzlibrary{positioning,shadows,arrows}

\begin{document}
\begin{center}
\begin{tikzpicture}[
    fact/.style={rectangle, draw=none, rounded corners=1mm, fill=blue, drop shadow,
        text centered, anchor=north, text=white},
    state/.style={rectangle, draw=none, fill=lightgray, text centered, anchor=north, text=black},
    leaf/.style={circle, draw=none, fill=red, circular drop shadow,
        text centered, anchor=north, text=white},
    level distance=1cm, growth parent anchor=south, node distance=2cm
]
\node [state] {$\mathcal{K}_0$} [->]
    child{
		node [state] {$S: = S, x \approx \epsilon$}
		child{
			node [state] {$\texttt{unsat}$}
		}
	}
	child{
		node [state] {$A: = A, \texttt{len} x > 0$}
		child{
			node [state] {$S: = S, y \approx \epsilon$}
			child{
				node [state] {$\texttt{unsat}$}
				}
		}
	}    
;
        
\end{tikzpicture}
\end{center}
\end{document}